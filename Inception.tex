\documentclass[11pt]{article}
\usepackage[margin=0.7in]{geometry} %reduce margins
\usepackage{comment} % allows block comments
\usepackage{color} %colored text
\usepackage{titling} % compresses the title
\usepackage[none]{hyphenat} %surround text in {} to prevent hyphenation
\usepackage{multirow}
\usepackage{array}

%squeezed itemize for table.
\newenvironment{packed_itemize}{
\begin{itemize}
 % \setlength{}{0pt}
  \setlength{\itemsep}{1pt}
  \setlength{\parskip}{0pt}
  \setlength{\parsep}{0pt}
}{\end{itemize}}

\author{Andrew Karnani, Auston Sterling, Jeff Rodowicz, and Vera Axelrod}
\title{YAExS {(Yet Another Exam Scheduler)}: Inception}
\date{September 24, 2012}

%Set up the title display
\pretitle{ \noindent\Large\bfseries}
\posttitle{\vspace{0.1in}\\}
\preauthor{\Large}
\postauthor{\vspace{0.05in}\\}
\predate{}
\postdate{}
\setlength{\droptitle}{-0.5in}


\begin{document}
\maketitle
\vspace{-0.1in}
\hrule

\tableofcontents % need to run latex twice to make this update properly

\vspace{0.3in}
\hrule

%%%%%%%%%%%%%%%%%%%%%%%%%
%%%%%%%%%%%%%%%%%%%%%%%%%
%%%%%%%%%%%%%%%%%%%%%%%%%
\section{Vision Statement}
\subsection{Executive Summary} %Vera

For years, final exam scheduling at Rensselaer Polytechnic Institute has been
 completed by hand in the Registrar’s Office. 
Unfortunately, this often leads to large numbers of students having three or more
 exams in one day and, in some cases, students with multiple exams scheduled at
 the same time.  
 The process is also time-consuming since it requires integration of many different
 types of data (course schedules, student registration, room availability, etc). 

We will create an automated web-based system to schedule exams which we call YAExS 
(Yet Another Exam Scheduler - pronounced ``yikes!").
This system will integrate all the data necessary to schedule exams to simplify the 
scheduling process, and will produce exam schedules that minimizes conflicts 
while respecting instructor and Registrar preferences whenever possible.  
The entire system will be customizable, easy to use, and flexible 
so that the exam schedule can be changed to satisfy last-minute requirements.

Using the Unified Process of software engineering, we will develop YAExS
 over the coming semester. 
This report represents the first stage of our project: Inception. 
Yet to come are the Project Elaboration, two Project Stakeholder Product Reviews,
 and a Transition Report.


%%%%%%%%%%%%%%%%%%%%%%%%%
%%%%%%%%%%%%%%%%%%%%%%%%%
%%%%%%%%%%%%%%%%%%%%%%%%%
\subsection{Business Case} %Andrew
\textcolor{blue}{ business case for the project (including a short analysis of the
 competition, market space or similar commercial off the shelf software) (12 points),}
, reduce/eliminate conflicts which cause headaches for students\&professors\&the registrar, minimize hand-word needed and so reduce errors,will speed up exam scheduling by allow preferences to be stated before the exam schedule is made so the schedule does not have to be remade several times, will allow for easy sensitivity analysis (e.g. what if we want to have just 16 exam periods, or what if we don't use west hall, or what if we want to minimize the number of students with back-to-back finals).
\subsection{Project Stakeholders} %Andrew
\textcolor{blue}{a description of the project stakeholders (7 points),}


%%%%%%%%%%%%%%%%%%%%%%%%%
%%%%%%%%%%%%%%%%%%%%%%%%%
%%%%%%%%%%%%%%%%%%%%%%%%%
\subsection{Major Features} %Auston
\textcolor{blue}{ list of the major features of the completed project (7 points),}
% General list for now. Formatting/Table-ing could be improved later
\begin{itemize}
\item Interface for course instructors
  \begin{itemize}
  \item Specify which courses have exams
  \item Request particular times, dates, rooms, or other preferences
  \item Receive updates on scheduling from the registrar
  \end{itemize}
\item Interface for the registrar
  \begin{itemize}
  \item Ability to start the scheduling program
  \item Communicate with course instructors and department schedulers
  \item Track which instructors have not yet responded
  \item Post the completed schedule to a public website
  \item Upload data on which courses each student is taking % Good for now? We still don't know the details here...
  \item Manual control over schedules after generation
  \end{itemize}
\item Interface for students
  \begin{itemize}
  \item View entire up-to-date exam schedule on website
  \end{itemize}
\end{itemize}

%%%%%%%%%%%%%%%%%%%%%%%%%
%%%%%%%%%%%%%%%%%%%%%%%%%
\subsection{Major Risks} %Auston

\begin{itemize}
\item The scheduling algorithm may not converge, leaving the core of the system useless.
\item Limited experience with web development and database management.
\item For privacy reasons, the Registrar may not be willing to provide detailed information about which courses students are taking. Without that, the scheduler would not be able to accurately prevent conflicts.
\item Some of our preferred programming languages may have difficulty communicating between each other. % Not sure about this one.
\end{itemize}


%%%%%%%%%%%%%%%%%%%%%%%%%
%%%%%%%%%%%%%%%%%%%%%%%%%
%%%%%%%%%%%%%%%%%%%%%%%%%
\section{User Scenarios}  %JEFF+VERA
\textcolor{blue}{The User Scenario should create a realistic picture of a persona using the software.  
Look for details about the Persona and the User Interface.  
Did the Persona achieve their goal?  
Was the User Scenario compelling?  
Is the picture painted realistic and detailed?
Each has a persona, goals, and sequence of steps}
\subsection{Instructors} %JEFF
\begin{description}
\item[Name:]
\item[Age:]
\item[Occupation:]
\item[Background:]

\item[Goals:]
\begin{enumerate}
\end{enumerate}

\item[Sequence of Steps:]
\end{description}


\subsection{Registrar}  %VERA

\begin{description}
\item[Name:] Reggie Star
\item[Age:]  65
\item[Occupation:] RPI Assistant Registrar
\item[Background:] Reggie came to RPI in 1984 when he  back moved from
 Chicago to raise his children in the capital area where he was raised. 
He worked in various offices of the administration before becoming
 the Assistant Registrar in 1997.  As the Assistant Registrar, he is in charge of planning RPI's academic calendar,  scheduling classes, overseeing student course registration and scheduling final exams.

He has used web-based applications in the past, but it takes him some 
time to get accustomed to new interfaces. 
He uses Microsoft Excel often but has never written his own code
 and has never used Excel’s optimization capabilities.

\item[Goals:]
Reggie would like to:
\begin{enumerate}
\item Reduce the number of make-up exams for students with two exams scheduled at the same time or with multiple exams in one day.
\item Speed up the exam scheduling process.
\item Keep his email inbox clear of schedule change requests from RPI instructors.
\item Reduce the effort he spends supervising work study students handcrafting lists of cross-listed courses at the beginning of each semester.
\item Minimize errors made by those work study students.
\item Update final exam schedule to accommodate last minute requests 
from professors.
\end{enumerate}

\item[Sequence of Steps]
During the second week of classes Reggie logs on to his YAExS account and
 sees that all instructors have already logged in and input their
 exam preferences.  
Reggie selects “instructor preference summary”  to see an overview of
 the data from instructors. He sees how many courses need to have exams
 scheduled, and how many of those are made up of multiple sections. 
He sees that a number of professors have indicated that they do not
want exams on the last day, and a few would prefer not to have evening exams.

Next, Reggie inputs his configuration settings and schedule exam.
He marks his preference as "reduce evening exams" since he has had many professors
in the past complain when their exams were scheduled in the evening, 
even if they didn't express that preference ahead of time. 
He then prompts the system to schedule exams.

The next day, Reggie returns to check on his exam schedule. 
When he logs in, he sees that an exam schedule has been computed. 
He chooses "exam schedule summary" and sees that
the the exam schedule has no exam conflicts, uses few evening slots, 
and respects all the professor preferences. 

Satisfied with this schedule, Reggie clicks: download exam schedule file
 and sends it to all department heads for a final review.


\end{description}

%%%%%%%%%%%%%%%%%%%%%%%%
%%%%%%%%%%%%%%%%%%%%%%%%
%%%%%%%%%%%%%%%%%%%%%%%%
\section{Project Schedule} %VERA

\centering % center the table
\begin{tabular}{|m{0.9in}|m{0.9in}|m{3.2in}|m{.9in}|}
\hline
\textbf{Phases}  & \textbf{Iterations}  & \textbf{Tasks}        & \textbf{Milestones} \\
\hline\hline
%%%%%%%%%%%%%%%%%%%%
 Inception 09/06 - 09/24
   &  Iteration I 09/06 - 09/24  
	& Inception Deliverables:
	 \begin{packed_itemize} 
	\vspace{-0.15in}
		\item Vision Statement
		\item Use Scenarios
		\item Project Schedule
	\end{packed_itemize}
	\vspace{-0.4in}
	& Inception Complete\\
   &        & & \\
\hline
%%%%%%%%%%%%%%%%%%%%
Elaboration 09/24-10/11& 
Iteration II 09/24 - 10/01& 
Elaboration Deliverables: 
	 \begin{packed_itemize} 
	\vspace{-0.15in}
		\item Domain Model Diagram
		\item Supplemental Specification
		\item Deployment Diagram
		\item Use Cases
		\item Work Breakdown Structure
		\item Updated Project Schedule
   \end{packed_itemize}

\raggedright{
Instructor and Registrar User Interface Prototype
}
\vspace{0.1in}

Collect Sample Data from Registrar
& Elaboration Complete
\\
& & & \\
\hline

%%%%%%%%%%%%%%%%%%%%
\multirow{10}{*}{Construction }
 &
 Iteration III 10/01 - 10/12 &
 Design of System:
	\begin{packed_itemize}
		\vspace{-0.15in}
		\item Static Class Diagram
		\item Design Approach
		\item Select the Optimization Software
	\end{packed_itemize}

 Design of Functions:
	\begin{packed_itemize}
		\vspace{-0.15in}
		\item Sequence Diagram
		\item  Design Pattern
	\end{packed_itemize}

 Implementation \# 1
	\begin{packed_itemize}
		\vspace{-0.15in}
		\item Code Skeleton
		\item Major Classes
	\end{packed_itemize}
 & Design Complete \\  \cline{2-4}
%%%%%
&
 Iteration IV 10/12 - 10/29 &
 Implementation \# 2: 
\emph{Must} features:
	\begin{packed_itemize}
		\item Database 
		\item Exam schedule creation
		\item Interface to display exam schedule
		\item 
	\end{packed_itemize}

 
Sprint \# 1 Deliverable: Object Calisthenics Sample Code
 & Iterative Release \\  \cline{2-4}
%%%%%
 &
 Iteration IV 10/29 - 11/19 &
 Implementation \# 3: Features to be added, including \textcolor{red}{additional feautres}

Spring \# 2 Deliverables: Testing Documents, Code Review &
 Stakeholder Review \#1 
and

Beta Release \\ \hline
%%%%%%%%%%%%%%%%%%%%
Transition  11/19 - 12/06 &
Iteration V 11/19 - 12/06 &
Implementation \#4: 
	\begin{packed_itemize}
		\vspace{-0.15in}
		\item Coordination of features
		\item Integrate entire system
	\end{packed_itemize}
Final Test Results

Final Deliverables:
	\begin{packed_itemize}
	\vspace{-0.15in}
		\item Evidence of Best Practices
		\item Peer Reviews
	\end{packed_itemize}
	\vspace{-0.4in}
& 
Stakeholder Review 2 and Final Release
\\
&&& \\
\hline
\end{tabular}



%%%%%%%%%%%%%%%%%%%%
%%%%%%%%%%%%%%%%%%%%
%%%%%%%%%%%%%%%%%%%%
\section{Contribution Summary} %EVERYONE
%\begin{table}
%\centering % center the table
\begin{tabular}{|m{1.4in}|m{4in}|}
\hline
\textbf{\large Name}     & \textbf{\large Contributions} \\
\hline\hline
%%%%%%%%%%%%%%%%%%%%
 Andrew Karnani
	& 
	 \begin{packed_itemize} 
		\item something
	\end{packed_itemize}
\\
\hline
 Auston Sterling
	& 
	 \begin{packed_itemize} 
	        \item Wrote Major Features
                \item Wrote Major Risks
                \item Wrote Status Report
	\end{packed_itemize}
\\
\hline
Jeffrey Rodowicz
	& 
	 \begin{packed_itemize} 
		\item something
	\end{packed_itemize}
\\
\hline
Vera Axelrod
	& 
	 \begin{packed_itemize} 
		\item Wrote Executive Summary 
		\item Wrote Reggie Star Use Scenario
		\item Wrote Project Schedule
		\item Coordinated meetings with Assistant Registrar
		\item Contributed research on competition
	\end{packed_itemize}
\\
\hline
\end{tabular}
%\end{table}


\section{Status Report} %AUSTON
\textcolor{blue}{Combine the weekly status updates}
\subsection{Things We've Done}
\begin{itemize}
\item Created a Github repository
\item Met with the Assistant Registrar
\item Set up a weekly meeting time
\item Completed Inception deliverables
\end{itemize}

\subsection{Challenges} % I'm using this for more immediate risks, so it won't be exactly the same as the Major Risks section
\begin{itemize}
\item Lack of experience with web development and database management
\item Available data on courses may be insufficient for proper optimization
\end{itemize}

\subsection{Upcoming Plans}
\begin{itemize}
\item Begin work on Elaboration deliverables
\item Plan coding strategy and distribute work
\item Stay in touch with Assistant Registrar
\end{itemize}
\end{document}
