\documentclass[11pt]{article}
\usepackage[margin=0.7in]{geometry} %reduce margins
\usepackage{comment} % allows block comments
\usepackage{color} %colored text
\usepackage{titling} % compresses the title
\usepackage[none]{hyphenat} %surround text in {} to prevent hyphenation
\usepackage{multirow}
\usepackage{array}

%squeezed itemize for table.
\newenvironment{packed_itemize}{
\begin{itemize}
 % \setlength{}{0pt}
  \setlength{\itemsep}{1pt}
  \setlength{\parskip}{0pt}
  \setlength{\parsep}{0pt}
}{\end{itemize}}

\author{Andrew Karnani, Auston Sterling, Jeff Rodowicz, and Vera Axelrod}
\title{YAExS {(Yet Another Exam Scheduler)}: Inception}
\date{September 24, 2012}

%Set up the title display
\pretitle{ \noindent\Large\bfseries}
\posttitle{\vspace{0.1in}\\}
\preauthor{\Large}
\postauthor{\vspace{0.05in}\\}
\predate{}
\postdate{}
\setlength{\droptitle}{-0.5in}


\begin{document}
\maketitle
\vspace{-0.1in}
\hrule

\tableofcontents % need to run latex twice to make this update properly

\vspace{0.3in}
\hrule

%%%%%%%%%%%%%%%%%%%%%%%%%
%%%%%%%%%%%%%%%%%%%%%%%%%
%%%%%%%%%%%%%%%%%%%%%%%%%
\section{Vision Statement}
\subsection{Executive Summary} %Vera

For years, final exam scheduling at Rensselaer Polytechnic Institute has been
 completed by hand in the Registrar’s Office.
Unfortunately, this often leads to large numbers of students having three or more
 exams in one day and, in some cases, students with multiple exams scheduled at
 the same time.
 The process is also time-consuming since it requires integration of many different
 types of data (course schedules, student registration, room availability, etc).

We will create an automated web-based system to schedule exams which we call YAExS
(Yet Another Exam Scheduler - pronounced ``yikes!").
This system will integrate all the data necessary to schedule exams to simplify the
scheduling process, and will produce exam schedules that minimizes conflicts
while respecting instructor and Registrar preferences whenever possible.
The entire system will be customizable, easy to use, and flexible
so that the exam schedule can be changed to satisfy last-minute requirements.

Using the Unified Process of software engineering, we will develop YAExS
 over the coming semester.
This report represents the first stage of our project: Inception.
Yet to come are the Project Elaboration, two Project Stakeholder Product Reviews,
 and a Transition Report.


%%%%%%%%%%%%%%%%%%%%%%%%%
%%%%%%%%%%%%%%%%%%%%%%%%%
%%%%%%%%%%%%%%%%%%%%%%%%%
\subsection{Business Case} %Andrew
\textcolor{blue}{ business case for the project (including a short analysis of the
 competition, market space or similar commercial off the shelf software) (12 points)}

\subsubsection*{The Problem}
Computing power has improved significantly over the last decade.  Computers are now powerful and flexible enough to solve many problems that previously could only be solved manually by humans.  For example generating directions from a map was previously a job more easily done by people, but today Google maps and personal GPS devices quickly and accurately plan routes between locations separated by five miles or five thousand.  Technology has also evolved enough to allow rapid gathering and dissemination of information via networking.  A letter which used to take 3-5 days to send now takes  3-5 seconds via e-mail.

\par Presently at RPI, final exam scheduling is done manually by several individuals in the Registrars office including the Assistant Registrar Michael Conroy.  The process involves emailing all faculty on campus to determine which classes/sections actually have final exams, and which classes/sections meet together for their exams.  After gathering all this information, the exam schedule is created.  Currently that process involves much copy and pasting in Excel and even more guess and check work.  After determining a preliminary schedule, rooms are assigned for each exam and the schedule is forwarded to faculty for a final check before being released to the student body.  Right now the Office of the Registrar relies on the students in the Polytechnic student newspaper to help catch conflicts.  Not only is the entire process inefficient but it typically results in a schedule where multiple students have conflicts or more than two exams in a day.  Per RPI policy, any student who is scheduled for more than two exams in one day (and of course for two exams at one time) is eligible to have a make-up exam. Scheduling make-up exams is a headache for the Registrar's Office, the course instructors, and the students, and can lead to unfair exam practices.

\subsubsection*{Project Feasibility}
\par We believe that using modern computers is possible to streamline this process using a single integrated IT solution.  All of the course and student registration data is already available electronically to the Registrar's Office on the banner system, all that remains in terms of data processing is to collect information from instructors about which exams need to be scheduled.

Once all the data is integrated into a single system, exams can be scheduled to reduce conflicts using mathematical optimization. Today dozens of commercial and open source optimizers are available that can easily handle a problem of this size on a standard RPI laptop in a matter of hours.

Currently the Registrar schedules exams over 20 periods (four time periods per day over five days).  Since there are fewer than 20 non-overlapping periods when the vast majority of RPI classes meet (e.g. Monday/Thursday from 8am-10am) and since the vast majority of students are enrolled in non-overlapping classes, finding an exam schedule with few conflicts should be possible.  In fact, previous work by Axelrod and Sebastian found exam schedules with no conflicts for RPI's Fall 2011 semester.

\subsubsection*{Competition}

Mitchell and Max: their program has major issues (see Axelrod and Sebastian).

Lots of paid things, too expensive!

open source projects may be hard to adapt to RPI, or difficult to integrate seamlessly into the RPI system.



\subsection{Project Stakeholders} %Andrew
\textcolor{blue}{a description of the project stakeholders (7 points)}
Team members

Michael Conroy

Team instructors

%%%%%%%%%%%%%%%%%%%%%%%%%
%%%%%%%%%%%%%%%%%%%%%%%%%
%%%%%%%%%%%%%%%%%%%%%%%%%
\subsection{Major Features} %Auston
\begin{table}[htbp]
  \centering
  \begin{tabular} {| l | r |}
    \hline
    \textbf{Feature} & \textbf{Priority} \\ \hline \hline
    \multicolumn{2}{|c|}{\textbf{Interface for course instructors}} \\ \hline
    Specify which courses have exams & Must \\ \hline
    Request particular times, dates, rooms, or other preferences & Should \\ \hline
    Receive updates on scheduling changes & Should \\ \hline
    \multicolumn{2}{|c|}{\textbf{Interface for the registrar}} \\ \hline
    Ability to start the scheduling program & Must \\ \hline
    Manual control over schedules after generation & Must \\ \hline
    Post the completed schedule to a public website & Should \\ \hline 
    Set scheduler preferences & Should \\ \hline
    Notify course instructors and department schedulers about deadlines and updates & Could \\ \hline
    Track which instructors have not yet responded & Could \\ \hline
    Provide student course registration information & Could \\ \hline %Vera: what is this?
    \multicolumn{2}{|c|}{\textbf{Interface for students}} \\ \hline
    View entire up-to-date schedule on website & Could \\ \hline
  \end{tabular}
  \caption{Table of major features, categorized by priority as Must, Should, or Could.}
\end{table}

%%%%%%%%%%%%%%%%%%%%%%%%%
%%%%%%%%%%%%%%%%%%%%%%%%%
\subsection{Major Risks} %Auston

\begin{itemize}
\item The scheduling algorithm may not converge, or it may somehow find a worse schedule than the Registar would by hand, leaving the core of the system useless.
\item Limited team member experience with web development and database management.
\item For privacy reasons, the Registrar may not be willing to provide detailed information about which courses students are taking. Without that, the scheduler would not be able to accurately prevent conflicts.
\item Some of our preferred programming languages and tools may have difficulty communicating between each other. % Not sure about this one.
\end{itemize}


%%%%%%%%%%%%%%%%%%%%%%%%%
%%%%%%%%%%%%%%%%%%%%%%%%%
%%%%%%%%%%%%%%%%%%%%%%%%%
\section{User Scenarios}  %JEFF+VERA
\textcolor{blue}{The User Scenario should create a realistic picture of a persona using the software.
Look for details about the Persona and the User Interface.
Did the Persona achieve their goal?
Was the User Scenario compelling?
Is the picture painted realistic and detailed?
Each has a persona, goals, and sequence of steps}
\subsection{Instructor} %JEFF
\begin{description}
\item[Name:]
\item[Age:]
\item[Occupation:]
\item[Background:]

\item[Goals:]
\begin{enumerate}
\end{enumerate}

\item[Sequence of Steps:]
\end{description}


\subsection{Registrar}  %VERA

\begin{description}
\item[Name:] Reggie Star
\item[Age:]  65
\item[Occupation:] RPI Assistant Registrar
\item[Background:] Reggie came to RPI in 1984 when he  back moved from
 Chicago to raise his children in the capital area where he was raised.
He worked in various offices of the administration before becoming
 the Assistant Registrar in 1997.  As the Assistant Registrar, he is in charge of planning RPI's academic calendar,  scheduling classes, overseeing student course registration and scheduling final exams.

He has used web-based applications in the past, but it takes him some
time to get accustomed to new interfaces.
He uses Microsoft Excel often but has never written his own code
 and has never used Excel’s optimization capabilities.

\item[Goals:]
Reggie would like to:
\begin{enumerate}
\item Reduce the number of make-up exams for students with two exams scheduled at the same time or with multiple exams in one day.
\item Speed up the exam scheduling process.
\item Keep his email inbox clear of schedule change requests from RPI instructors.
\item Reduce the effort he spends supervising work study students handcrafting lists of cross-listed courses at the beginning of each semester.
\item Minimize errors made by those work study students.
\item Update final exam schedule to accommodate last minute requests
from professors.
\end{enumerate}

\item[Sequence of Steps]
During the second week of classes Reggie logs on to his YAExS account and
 sees that all instructors have already logged in and input their
 exam preferences.
Reggie selects “instructor preference summary”  to see an overview of
 the data from instructors. He sees how many courses need to have exams
 scheduled, and how many of those are made up of multiple sections.
He sees that a number of professors have indicated that they do not
want exams on the last day, and a few would prefer not to have evening exams.

Next, Reggie inputs his configuration settings and schedule exam.
He marks his preference as "reduce evening exams" since he has had many professors
in the past complain when their exams were scheduled in the evening,
even if they didn't express that preference ahead of time.
He then prompts the system to schedule exams.

The next day, Reggie returns to check on his exam schedule.
When he logs in, he sees that an exam schedule has been computed.
He chooses "exam schedule summary" and sees that
the the exam schedule has no exam conflicts, uses few evening slots,
and respects all the professor preferences.

Satisfied with this schedule, Reggie downloads the exam schedule
 and sends it to all department heads for a final review.


\end{description}

%%%%%%%%%%%%%%%%%%%%%%%%
%%%%%%%%%%%%%%%%%%%%%%%%
%%%%%%%%%%%%%%%%%%%%%%%%
\section{Project Schedule} %VERA

\begin{tabular}{|m{0.9in}|m{0.9in}|m{4in}|m{.8in}|}
\hline
\textbf{Phases}  & \textbf{Iterations}  & \textbf{Tasks}        & \textbf{Milestones} \\
\hline\hline
%%%%%%%%%%%%%%%%%%%%
Inception 09/06 - 09/24 &
Iteration I 09/06 - 09/24 & \vspace{0.1in}
Inception Deliverables:
	 \begin{packed_itemize}
	\vspace{-0.15in}
		\item Vision Statement
		\item Use Scenarios
		\item Project Schedule
	\vspace{-0.15in}
	\end{packed_itemize}
	& Inception Complete\\
\hline
%%%%%%%%%%%%%%%%%%%%
Elaboration 09/24-10/11&
Iteration II 09/24 - 10/01&  \vspace{0.1in}
Elaboration Deliverables:
	 \begin{packed_itemize}
	\vspace{-0.15in}
		\item Domain Model Diagram
		\item Supplemental Specification
		\item Deployment Diagram
		\item Use Cases
		\item Work Breakdown Structure
		\item Updated Project Schedule
   \end{packed_itemize}

\raggedright{
Instructor and Registrar User Interface Prototype
}

Collect Sample Data from Registrar
& Elaboration Complete
\\
\hline

%%%%%%%%%%%%%%%%%%%%
\multirow{10}{*}{Construction }
 &
 Iteration III 10/01 - 10/12 & \vspace{0.1in}
 Design of System:
	\begin{packed_itemize}
		\vspace{-0.15in}
		\item Static Class Diagram
		\item Design Approach
		\item Select the Optimization Software
	\end{packed_itemize}

 Design of Functions:
	\begin{packed_itemize}
		\vspace{-0.15in}
		\item Sequence Diagram
		\item  Design Pattern
	\end{packed_itemize}

 Implementation \# 1
	\begin{packed_itemize}
	\vspace{-0.15in}
		\item Code and Database Skeleton
		\item Major Classes
	\vspace{-0.15in}
	\end{packed_itemize}
 & Design Complete \\  \cline{2-4}
%%%%%
&
 Iteration IV 10/12 - 10/29 & \vspace{0.1in}
 Implementation \# 2:
\emph{Must} features:
	\begin{packed_itemize}
	\vspace{-0.15in}
		\item Cross-reference course detection
		\item Schedule exams and assign rooms
		\item Interface to display exam schedule
	\end{packed_itemize}

{\raggedright
Sprint \# 1 Deliverable:
Object Calisthenics Sample Code }
 & Iterative Release \\  \cline{2-4}
%%%%%
 &
 Iteration IV 10/29 - 11/19 & \vspace{0.1in}
 Implementation \# 3: Additional Features
	\begin{packed_itemize}
	\vspace{-0.15in}
		\item Instructor exam sign-up website
		\item Registrar exam schedule creation website
		\item Warm start the exam scheduling process
	\end{packed_itemize}

Sprint \# 2 Deliverables:
 Testing Documents, Code Review &
Stakeholder Review \#1
and
Beta Release \\ \hline
%%%%%%%%%%%%%%%%%%%%
Transition  11/19 - 12/06 &
Iteration V 11/19 - 12/06 & \vspace{0.1in}
Implementation \#4:
	\begin{packed_itemize}
		\vspace{-0.15in}
		\item Coordination of features
		\item Integrate entire system
	\end{packed_itemize}
Final Test Results

Final Deliverables:
	\begin{packed_itemize}
	\vspace{-0.15in}
		\item Evidence of Best Practices
		\item Peer Reviews
	\vspace{-0.15in}
	\end{packed_itemize}
&
Stakeholder Review 2 and Final Release \\
\hline
\end{tabular}


%%%%%%%%%%%%%%%%%%%%
%%%%%%%%%%%%%%%%%%%%
%%%%%%%%%%%%%%%%%%%%
\section{Contribution Summary} %EVERYONE
%\begin{table}
%\centering % center the table
\begin{tabular}{|m{1.4in}|m{4in}|}
\hline
\textbf{\large Name}     & \textbf{\large Contributions} \\
\hline\hline
%%%%%%%%%%%%%%%%%%%%
 Andrew Karnani
	&
	 \begin{packed_itemize}
		\item something
	\end{packed_itemize}
\\
\hline
 Auston Sterling
	&
	 \begin{packed_itemize}
	        \item Wrote Major Features
                \item Wrote Major Risks
                \item Wrote Status Report
	\end{packed_itemize}
\\
\hline
Jeffrey Rodowicz
	&
	 \begin{packed_itemize}
		\item something
	\end{packed_itemize}
\\
\hline
Vera Axelrod
	&
	 \begin{packed_itemize}
		\item Wrote Executive Summary
		\item Wrote Reggie Star Use Scenario
		\item Wrote Project Schedule
		\item Coordinated meetings with Assistant Registrar
		\item Contributed research on competition
	\end{packed_itemize}
\\
\hline
\end{tabular}
%\end{table}


\section{Status Report} %AUSTON
\textcolor{blue}{Combine the weekly status updates}
\subsection{Things We've Done}
\begin{itemize}
\item Created a Github repository and gained experience with Github in writing the Inception deliverables
\item Met with the Assistant Registrar
\item Set up a weekly meeting time and reserved a room on campus
\item Chosen a project name
\item Completed Inception deliverables
\end{itemize}

\subsection{Immediate Challenges} % I'm using this for more immediate risks, so it won't be exactly the same as the Major Risks section
\begin{itemize}
\item Lack of experience with web development and database management
\item Available data on courses may be insufficient or difficult to find
\end{itemize}

\subsection{Upcoming Plans}
\begin{itemize}
\item Begin work on Elaboration deliverables
\item Plan coding strategy and distribute work
\item Select a mathematical optimization package/framework
\item Stay in touch with Assistant Registrar
\end{itemize}

\section*{References}

\hspace{0.25in}Axelrod, Vera and Katie Sebastian. Exam Scheduling. May 17, 2012. Unpublished project report for MATP 6640/ISYE 6770 Linear Programming.

McGinley,  Patton. 2011 Linear Programming Software Survey. ORMS Today, Vol. 38, No. 3.  Accessed Sep 22, 2012. <http://www.orms-today.org/surveys/LP/LP-survey.html>

\end{document}


