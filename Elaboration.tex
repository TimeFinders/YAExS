\documentclass[11pt]{article}
\usepackage[margin=1in]{geometry}
\usepackage{graphicx} %figures
\usepackage{titling}
\usepackage{color}
\usepackage{comment}
\usepackage[none]{hyphenat} %surround text in {} to prevent hyphenation
\usepackage{multirow} %helps tables
\usepackage{array} %helps tables
\usepackage{url}

%squeezed itemize for table.
\newenvironment{packed_itemize}{
\begin{itemize}
 % \setlength{}{0pt}
  \setlength{\itemsep}{1pt}
  \setlength{\parskip}{0pt}
  \setlength{\parsep}{0pt}
}{\end{itemize}}


\title{YAExs Elaboration}
\author{TimeFinders: Andrew Karnani, Auston Sterling, Jeffrey Rodowicz and Vera Axelrod}
\date{October 11, 2012}

\begin{document}
\maketitle
\tableofcontents
\vspace{0.2in}
\hrule
\vspace{1in}

%%%%%%%%%%%%%%%%%%%%%%%%%%%%%%%%%%%%%%%%%%%
%%%%%%%%%%%%%%%%%%%%%%%%%%%%%%%%%%%%%%%%%%%
\section{Domain Model}
Diagram

%Our domain model diagram is shown in Figure \ref{fig:Domain}.

% To include the figure just save the Domain Model diagram in the same directory
% as this tex file, then change {DomainModel.png} to the filename.
% Reference this figure in the text as \ref{fig:Domain}
\begin{comment}
\begin{figure}
	\centering
		\includegraphics[width = \textwidth]{DomainModel.png}
	\caption{Domain Model}
	\label{fig:Domain}
\end{figure}
\end{comment}

%%%%%%%%%%%%%%%%%%%%%%%%%%%%%%%%%%%%%%%%%%%
%%%%%%%%%%%%%%%%%%%%%%%%%%%%%%%%%%%%%%%%%%%
\section{Supplemental Specification}


\begin{tabular}{|m{1in}|m{0.3in}|m{0.6in}|m{4.5in}|}
\hline
\textbf{Category}  & \textbf{ID}  & \textbf{Priority}        & \textbf{Description} \\
\hline\hline
%%%%%%%%%%%%%%%%%%%%
\multirow{3}{*}{Functionality }
 & F01 & Must
 & Description \\  \cline{2-4}
%%%%%
 & F02 & Should
 & Description \\  \cline{2-4}
%%%%%
 & F03 & Could
 & Description \\ \hline

%%%%%%%%%%%%%%%%%%%%
\multirow{3}{*}{Usability }
 & F04 & Must
 & Description \\  \cline{2-4}
%%%%%
 & F05 & Should
 & Description \\  \cline{2-4}
%%%%%
 & F06 & Could
 & Description \\ \hline

%%%%%%%%%%%%%%%%%%%%
\multirow{3}{*}{Reliability }
 & F07 & Must
 & Description \\  \cline{2-4}
%%%%%
 & F08 & Should
 & Description \\  \cline{2-4}
%%%%%
 & F09 & Could
 & Description \\ \hline


%%%%%%%%%%%%%%%%%%%%
\multirow{3}{*}{Performance }
 & F07 & Must
 & Description \\  \cline{2-4}
%%%%%
 & F08 & Should
 & Description \\  \cline{2-4}
%%%%%
 & F09 & Could
 & Description \\ \hline

%%%%%%%%%%%%%%%%%%%%
\multirow{3}{*}{Maintainability}
 & F07 & Must
 & Description \\  \cline{2-4}
%%%%%
 & F08 & Should
 & Description \\  \cline{2-4}
%%%%%
 & F09 & Could
 & Description \\ \hline

%%%%%%%%%%%%%%%%%%%%
\multirow{3}{*}{Configurability}
 & F07 & Must
 & Description \\  \cline{2-4}
%%%%%
 & F08 & Should
 & Description \\  \cline{2-4}
%%%%%
 & F09 & Could
 & Description \\ \hline
\end{tabular}

\textcolor{red}{Constraints?}

%%%%%%%%%%%%%%%%%%%%%%%%%%%%%%%%%%%%%%%%%%%
%%%%%%%%%%%%%%%%%%%%%%%%%%%%%%%%%%%%%%%%%%%
\section{Deployment}
Diagram

%%%%%%%%%%%%%%%%%%%%%%%%%%%%%%%%%%%%%%%%%%%
%%%%%%%%%%%%%%%%%%%%%%%%%%%%%%%%%%%%%%%%%%%
\section{Use Cases}


\subsection{Department Scheduler}
\begin{description}
\item[Name:] Debby Schedson
\item[Age:] 32
\item[Occupation:] Undergraduate Coordinator, PPI Mathematical Sciences
\item[Background:]  Debby  attended Sage College before becoming an administrative assistant,
and later taking on the role of Undergraduate Coordinator  of the RPI Math department.  She handles

..
...



\item[Goals:]
\begin{enumerate}
\item  
\item
\item
\end{enumerate}

\item[Sequence of Steps:]

\end{description}


\footnote{After conversing with RPI administrators Michael Conroy and Jeff Miner, we
decided to change our use case from individual professors to department schedulers.}

\subsection{Registrar}
\textcolor{red}{NEEDS TO BE MORE SPECIFIC}
\begin{description}
\item[Name:] Reggie Star
\item[Age:]  65
\item[Occupation:] Assistant Registrar, RPI 
\item[Background:] Reggie came to RPI in 1984 when he  back moved from
 Chicago to raise his children in the capital area where he was raised.
He worked in various offices of the administration before becoming
 the Assistant Registrar in 1997.  As the Assistant Registrar, he is in charge of planning RPI's academic calendar, scheduling classes, overseeing student course registration and scheduling final exams.

He has used web-based applications in the past, but it takes him some
time to get accustomed to new interfaces.
He uses Microsoft Excel often but has never written his own code
 and has never used Excel’s optimization capabilities.

\item[Goals:]
Reggie would like to:
\begin{enumerate}
\item Reduce the number of make-up exams for students with two exams scheduled at the same time or with multiple exams in one day.
\item Speed up the exam scheduling process.
\item Keep his email inbox clear of schedule change requests from RPI instructors.
\item Reduce the effort he spends supervising work study students handcrafting lists of cross-listed courses at the beginning of each semester.
\item Minimize errors made by those work study students.
\item Update final exam schedule to accommodate last minute requests
from professors.
\end{enumerate}

\item[Sequence of Steps]
During the second week of classes Reggie logs on to his YAExS account and
 sees that all instructors have already logged in and input their
 exam preferences.
Reggie selects “instructor preference summary” to see an overview of
 the data from instructors. He sees how many courses need to have exams
 scheduled, and how many of those are made up of multiple sections.
He sees that a number of professors have indicated that they do not
want exams on the last day, and a few would prefer not to have evening exams.

Next, Reggie inputs his configuration settings and schedule exam.
He marks his preference as "reduce evening exams" since he has had many professors
in the past complain when their exams were scheduled in the evening,
even if they didn't express that preference ahead of time.
He then prompts the system to schedule exams.

The next day, Reggie returns to check on his exam schedule.
When he logs in, he sees that an exam schedule has been computed.
He chooses "exam schedule summary" and sees that
the exam schedule has no exam conflicts, uses few evening slots,
and respects all the professor preferences.

Satisfied with this schedule, Reggie downloads the exam schedule
 and sends it to all department heads for a final review.


\end{description}


%%%%%%%%%%%%%%%%%%%%%%%%%%%%%%%%%%%%%%%%%%%
%%%%%%%%%%%%%%%%%%%%%%%%%%%%%%%%%%%%%%%%%%%
\section{Work Breakdown}




%%%%%%%%%%%%%%%%%%%%%%%%%%%%%%%%%%%%%%%%%%%
%%%%%%%%%%%%%%%%%%%%%%%%%%%%%%%%%%%%%%%%%%%
\section{Project Schedule}

\end{document}